\chapter*{Fluidity Primer for Red Hat Enterprise and derivatives}
% \pagenumbering{roman} \setcounter{page}{}

Please check the \fluidity\ webpage at
\href{http://amcg.ese.ic.ac.uk/Fluidity}{http://amcg.ese.ic.ac.uk/Fluidity}
to ensure you are reading the most recent version of this manual. Methods for
installing Fluidity may sometimes change, and instructions may be updated!

This is a one-page primer for obtaining \fluidity\ and running a simple example. It assumes that:

\begin{itemize}
 \item You are running Red Hat Enterprise Linux 6.x or a derivative such as CentOS 6.x
 \item You have the EPEL repository enabled
         \footnote{
           If you do not already have the EPEL repository enabled on your workstation, 
           download the relevant 'epel-release' package from
           \href{https://fedoraproject.org/wiki/EPEL}{https://fedoraproject.org/wiki/EPEL}
           and install it on your workstation. 
         }
         (and no other third-party repositories)
 \item You have administrative rights on your computer
 \item You know how to run a terminal with a command prompt
 \item You have a directory in which you can create files
\end{itemize}

Set up your computer to access the Fluidity repository by typing the following,
all on one line:

\begin{lstlisting}[language=Bash]
sudo yum-config-manager --add-repo
                http://fluidityproject.github.com/yum/fluidity-rhel6.repo
\end{lstlisting}

Type your password when prompted.

Once this completes, update your system and install \fluidity along with its
supporting software by typing:

\begin{lstlisting}[language=Bash]
sudo yum install fluidity
\end{lstlisting}

Now uncompress the packaged examples to a directory in which you can create
files (in this example, \lstinline[language=Bash]+/tmp+) by typing:

\begin{lstlisting}[language=Bash]
cd /tmp
tar -zxvf /usr/share/doc/fluidity/examples.tar.gz
\end{lstlisting}

Change into an examples directory (top\_hat is suggested as a straightforward
starter) and run the example:

\begin{lstlisting}[language=Bash]
cd examples/top_hat/
make preprocess
make run
make postprocess
\end{lstlisting}

You have now run your first \fluidity model. Chapter \ref{chap:examples}
describes this and the other examples provided with \fluidity.  
