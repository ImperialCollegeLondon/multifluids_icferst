\chapter{Useful numbers}

The table below gives some useful physical values for parameters often used in modelling.

\begin{center}
\begin{longtable}{lll}
\hline
Definition & Symbol & Value\\
\hline
\endfirsthead
%
\multicolumn{2}{c}{{\tablename} -- Continued} \\[0.5ex]
\hline
Definition & Symbol & Value and units\\
\hline
\endhead
%This is the footer for all pages except the last page of the table...
  \\[0.5ex]
  \multicolumn{2}{l}{{Continued on Next Page\ldots}} \\
\endfoot
%This is the footer for the last page of the table...
  \hline
\endlastfoot
%
Radius of Earth (at equator)                    &  $R_E^{eq}$   &  $\m[6.3781\times 10^6]$\\
Radius of Earth (at pole)                       &  $R_E^{p}$    &  $\m[6.3568\times 10^6]$\\
Radius of Earth (average value)                 &  $R_E^{av}$   &  $\m[6.371\times 10^6]$\\
Mass of Earth                                   &  $M_E$        &  $\kg[5.9742\times 10^{24}]$\\
Mass of Moon                                    &  $M_M$        &  $\kg[7.36\times 10^{22}]$\\
Mass of Sun                                     &  $M_S$        &  $\kg[1.98892\times 10^{30}]$\\
Earth's rotation rate (based on sidereal day)   &  $\Omega$     &  $\rads[7.2921\times 10^{-5}]$\\
\end{longtable}
\end{center}
