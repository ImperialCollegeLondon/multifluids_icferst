\chapter*{Overview}
% \pagenumbering{roman} \setcounter{page}{}

\fluidity\ is an open source, general purpose, multi-phase CFD code capable of solving numerically the Navier-Stokes and accompanying field equations on arbitrary unstructured finite element meshes in one, two and three dimensions. It uses a moving finite element/control volume method which allows arbitrary movement of the mesh with time dependent problems. It has a wide range of finite element/control volume element choices including mixed formulations. \fluidity is coupled to a mesh optimisation library allowing for dynamic mesh adaptivity and is parallelised using MPI.

Chapter \ref{chap:gettingstarted} of this manual gives details on how prospective users can obtain and set up \fluidity\ for use on a personal computer or laptop. The fluid and accompanying field equations solved by \fluidity\ and details of the numerical discretisations available are discussed in chapters \ref{chap:model_equations} and \ref{chap:numerical_discretisation} respectively. When discretising fluid domains in order to perform a numerical simulations it is inevitable that at certain scales the dynamics will not be resolved. These sub-grid scale dynamics can however play an important in the large scale dynamics the user wishes to resolve. It is therefore necessary to parameterise these sub-grid scale dynamics and details of the parameterisations available within \fluidity\ are given in chapter \ref{chap:parameterisations}. \fluidity\ also contains embedded models for the modelling of non-fluid processes. Currently, a simple biology (capable of simulating plankton ecosystems) and a sediments model are available and these models are detailed in chapter \ref{chap:embedded}.

As mentioned above, one of the key features of \fluidity\ is its ability to adaptively re-mesh to various fields so that resolution can be concentrated in regions where the user wishes to accurately resolve the dynamics. Details regarding the adaptive re-meshing and the manner in which \fluidity\ deals with meshes are given in chapters \ref{chap:Adaptivity} and \ref{chap:meshes}.

\fluidity\ has its own specifically designed options tree to make configuring simulations as painless as possible. This options tree can be viewed and edited using the diamond GUI. Details on how to configure the options tree are given in \ref{chap:configuration}. Output from simulations is in the VTK format and details regarding viewing and manipulating output files are given in chapter \ref{chap:visualisation_and_diagnostics}. Finally, in order to introduce users to a range of common configurations and to the functionality available within \fluidity, chapter \ref{chap:examples} presents examples covering a range of fluid dynamics problems. For information regarding the style and scope of this manual, please refer to Appendix \ref{App:about}.