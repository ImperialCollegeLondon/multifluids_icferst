\chapter{Mesh generation}\label{chap:meshes}
\index{mesh!generation}
\index{grid|see{mesh}}

This chapter describes how to create a mesh for your \fluidity\ simulation. There
are many tools available, both free and proprietary, that are capable of generating
unstructured meshes that are suitable for use with \fluidity. Here, we restrict ourselves
to the mesh generation capabilities built into \fluidity\ and the associated script and
tools, mentioning only briefly some of the more common tools also available.

\section{Mesh data}

A mesh describes the computational domain in which a simulation takes
place. Regardless of the mesh file format in use, the information conveyed
is essentially the same.

\subsection{Node location}

The locations of the element vertices are recorded. Usually, these have the
same dimension as the topological dimension of the mesh elements. \fluidity\
does not currently support configurations such as shells in which the node
location dimension differs from the element topology dimension.

\subsection{Element topology}

The mesh is composed of elements. In one dimension these will be intervals
with each interval joining two nodes. In two dimensions, triangles or
quadrilaterals are supported with the elements joining three or four nodes
respectively. In three dimensions, the elements can be tetrahedra or
hexahedra and will join four or eight nodes.

The element topology will store the indices of the nodes which make up
each of the elements in the mesh.

\subsection{Facets}

Facets form the surface of elements. In one dimension, the facets of an
element are its bounding nodes. In two dimensions, the facets are the edge
intervals while the facets of a three-dimensional tetrahedral element are
triangles and those of a hexahedral element are quadrilaterals. External
mesh formats tend to only supply facet topology information for facets on
the surface of each domain. For each facet specified, the node indices of
that facet will be given. These surface facets are used in combination with
surface IDs to specify the regions over which boundary conditions should be
applied.

\subsection{Surface IDs}\label{sect:surface_ids}
\index{surface ID}
Numbers can be assigned to label particular facets (boundary nodes, edges or
faces in 1, 2 or 3 dimensions respectively) in order to set
boundary conditions or other parameters. This number can then be used to
specify which surface a particular boundary condition should be applied to
in \fluidity. 

\subsection{Region IDs}\label{sect:region_ids}
\index{region ID} These are analogous to surface IDs, however they are
associated with elements rather than facets. Region IDs may be used in
\fluidity\ to specify different initial conditions or material properties to
different parts of the domain.



\section{The triangle format}\label{sect:triangle_format}
\index{mesh!file formats}

The main input file format for \fluidity\ meshes is the \emph{triangle} format.
The triangle format is a ASCII file format originally designed for 2D
triangle meshes, but it can be easily extended to different dimensions and
more complex geometries.  \fluidity\ supports a version of the triangle format
which supports 1D, 2D and 3D meshes.  The following table shows the supported
combinations of element dimension and geometry.

\begin{tabular}{ l l l }
\textbf{Dimension} & \textbf{Geometry} & \textbf{Number of vertices per element} \\ \hline
1D & Line & 2\\ 
2D & Triangles &  3 \\ 
2D & Quadrilateral\footnote{Limitation: at the moment, the domain must be rectangular} & 4 \\
3D & Tetrahedra & 4 \\ 
3D & Hexahedra\footnote{Limitation: at the moment, the domain must be a cuboid} & 8 \\
\end{tabular}

A complete triangle mesh consists of three files: one file defining the
nodes of the mesh, one file describing the elements (for example triangles
in 2D) and one file defining the boundary parts of the mesh.

The triangle file format is very simple. Since the data is stored in ASCII,
any text editor can be used to edit the files.  Lines starting with \# will
be interpreted as a comment by \fluidity.  The filename should end with
either .node, .ele, .bound, .edge or .face.  The structure of these files
will now be explained:

\begin{description}
\item[.node file]
This file holds the coordinates of the nodes. The file structure is:

First line
\begin{lstlisting}: 
<total number of vertices> <dimension (must be 1,2 or 3)> 0 0
\end{lstlisting}
Remaining lines
\begin{lstlisting}
<vertex number> <x> [<y> [<z>]]
\end{lstlisting} 
where x, y and z are the coordinates.

Vertices must be numbered consecutively, starting from one. 

\item[.ele file] Saves the elements of the mesh. The file structure is:
\index{region ID}

First line:
\begin{lstlisting}
 <total number of elements> <nodes per element>  1
\end{lstlisting}
Remaining lines:
\begin{lstlisting} 
<element number> <node> <node> <node> ... <region id>
\end{lstlisting}  
The elements must be numbered consecutively, starting from one. Nodes are
indices into the corresponding .node file. For example in case of describing
a 2D triangle mesh, the first three nodes are the corner vertices. The
region ID can be used by \fluidity\ to set conditions on different parts of
the mesh, see section \ref{sect:region_ids}.

\item[.bound file] This file is only generated for one-dimensional meshes.
  It records the boundary points and assigns surface IDs to them. The file
  structure is:\index{surface ID}

First line:
\begin{lstlisting}
 <total number of boundary points> 1
\end{lstlisting}
Remaining lines:
\begin{lstlisting} 
<boundary point number> <node> <surface id>
\end{lstlisting}  
The boundary points must be numbered consecutively, starting from one. Nodes
are indices into the corresponding .node file. The surface ID is used by
\fluidity\ to specify parts of the surface where different boundary
conditions will be applied, see section \ref{sect:surface_ids}.

\item[.edge file] This file is only generated for two-dimensional meshes.
  It records the edges and assigns surface IDs to part of the mesh surface.
  The file structure is:

First line:
\begin{lstlisting}
 <total number of edges> 1
\end{lstlisting}
Remaining lines:
\begin{lstlisting} 
<edge number> <node> <node> ... <surface id>
\end{lstlisting}  
The edges must be numbered consecutively, starting from one. Nodes are
indices into the corresponding .node file. The surface ID is used by
\fluidity\ to specify parts of the surface where different boundary
conditions will be applied, see section \ref{sect:surface_ids}.


\item[.face file] This file is only generated for three-dimensional meshes.
  It records the faces and assigns surface IDs to part of the mesh surface. The
  file structure is:

First line:
\begin{lstlisting}
 <total number of faces> 1
\end{lstlisting}
Remaining lines:
\begin{lstlisting} 
<face number> <node> <node> <node> ... <surface id>
\end{lstlisting}  
The faces must be numbered consecutively, starting from one. Nodes are
indices into the corresponding .node file. The surface ID is used by
\fluidity\ to specify parts of the surface where different boundary
conditions will be applied, see section \ref{sect:surface_ids}.
\end{description}

To clarify the file format, a simple 1D example is shown: 
The following .node file defines 6 equidistant nodes from 0.0 to 5.0
\begin{lstlisting}
# example.node
6 1 0 0
1 0.0
2 1.0
3 2.0
4 3.0
5 4.0
6 5.0
\end{lstlisting}
The .ele file connects these nodes to 5 lines. Two regions will be defined with the IDs 1 and 2.
\begin{lstlisting}
# example.ele
5 2 1
1 1 2 1
2 2 3 1
3 3 4 1
4 4 5 2
5 5 6 2
\end{lstlisting}
Finally, the .bound file tags the very left and very right nodes as boundary
points an assigns the surface IDs 1 and 2 to them.
\begin{lstlisting}
# example.bound
2 1
1 1 1
2 6 2
\end{lstlisting}

\section{The Gmsh format}\label{sect:gmsh_format}


\fluidity\ also contains  support for the Gmsh format. Gmsh is a mesh
generator freely available on the Web at (\url{http://geuz.org/gmsh/}), and 
is included in Linux distributions such as Ubuntu. 

Unlike triangle files, Gmsh meshes are contained within one file, which have
the extension \lstinline[language=bash]+.msh+. The file contents may
be either binary or ASCII.

\subsection{Using Gmsh files with \fluidity}\label{sect:using_gmsh}

There are two ways in which Gmsh files can be used with \fluidity. The first
is to run \lstinline[language=Bash]{gmsh2triangle} and import the resulting
triangle files; this works on ASCII Gmsh files only. The second way is to 
use Fluidity's native Gmsh support, which loads in Gmsh files directly into
\fluidity, and works with binary and ASCII Gmsh formats. To enable native 
support, \fluidity\ needs to be told to expect a Gmsh file, which is achieved 
by setting the \onlypdf\option{/geometry/mesh/from\_file/format}\ option 
to \onlypdf\option{gmsh}.  \fluidity\ will now look for a file with the extension 
\lstinline[language=bash]+.msh+ when it runs.

For parallel simulations, you must use \lstinline[language=bash]+fldecomp+ to decompose a Gmsh
mesh into sub-meshes for each process. Here, only binary Gmsh files can be
used - see section \ref{mesh!meshing tools!fldecomp}\ for details.

Using native Gmsh support within Fluidity can result in faster start-up times,
especially when using large meshes and binary Gmsh files. Whilst it works
reliably in many circumstances, it is still classed as experimental.  You are 
encouraged to try using it, but if you run into any problems you should
 default to using the
\lstinline[language=bash]+gmsh2triangle+ utility and triangle format
instead.

\subsection{Gmsh file format}\label{sect:gmsh_file_format}

This section briefly describes the Gmsh format, and is only intended
to serve as a short introduction. If you need further
information, please read the official Gmsh documentation
(\url{http://geuz.org/gmsh/doc/texinfo/gmsh.pdf}).
Typically Gmsh files used in \fluidity\ contain three parts: a header, a
section for nodes, and one for elements. These  are explained in more
detail below.



\subsubsection*{The header}\label{sect:gmsh_header_section}
This contains Gmsh file version information, and indicates whether 
the main data is in ASCII or binary format. This will typically look like:
\begin{lstlisting}
$MeshFormat
2.1 0 8
[Extra data here in binary mode]
$EndMeshFormat
\end{lstlisting}

From the listing above we can tell that:
\begin{itemize}
\item the Gmsh format version is 2.1
\item it is in ASCII, as indicated by the 0 (1=binary)
\item the byte size of double precision is 8
\end{itemize}
In addition, in binary mode the integer 1 is written as 4 raw bytes, to check that the endianness of the Gmsh file and the system are the same (\textit{you will rarely have to worry about this})



\subsubsection*{The nodes section}\label{sect:gmsh_nodes_section}

The next section contains node data, viz:
\begin{lstlisting}
$Nodes
number_of_nodes
[node data]
$EndNodes
\end{lstlisting}

The \lstinline+[node data]+ part contains the listing of nodes, with ID,
followed by $x$, $y$, and $z$ coordinates. This part
will be in binary when binary mode has been selected. Note that even with 2D
problems, there will be a zeroed \textit{z} coordinate.



\subsubsection*{The elements section}\label{sect:gmsh_elements_section}

The elements section contains information on both facets and regular
elements. It also varies between binary and ASCII formats. The ASCII version
is:

\begin{lstlisting}
$Elements
element1_id element_type number_of_tags tag_list node_number_list
element2_id ...
...
...
$EndElements
\end{lstlisting}
\textit{Tags} are integer properties ascribed to the element. In \fluidity,
usually we are only concerned with the first one, the physical ID. This can mean
one of two things:

\begin{itemize}
\item A region ID - in the case of elements
\item A surface ID - in the case of facets
\end{itemize}

Since Gmsh doesn't explicitly label facets or regular elements as such,
internally \fluidity\ works this out from type: eg, if there a mesh consists
of tetrahedra and triangles, then triangles must be the facets.

\subsubsection*{Internal use }
\label{sect:gmsh_internal_use_section}

\fluidity\ occasionally augments GMSH files for its own internal use. It
does this in two cases. Firstly, when generating periodic meshes 
through \lstinline+periodise+ (section \ref{sect:decomposing_meshes_periodise}), the
fourth element tag is reserved to label elements at the periodic boundary.
Secondly, when checkpointing extruded meshes, a new section called
\lstinline+$NodeData+ is created at the end of the GMSH file; this contains
essential node column ID information.
 

\section{Mesh types}
\subsection{Extruded meshes}
\label{sect:extruded_meshes}
\index{mesh!extruded}

Given a 1D or 2D mesh file in triangle format, \fluidity\ can extrude this
mesh to create a layered 2D or 3D mesh, on which simulations can be
performed. The extrusion is specified in the .flml file and is documented in
section \ref{Sect:extruded}.

\subsection{Periodic meshes}
\index{mesh!periodic} 
\index{periodic domain} 
\label{mesh!mesh types!periodic} 
Periodic meshes are those that are ``virtually'' connected in one or more directions. To make a periodic
mesh you must first create a triangle file where the edges that are periodic
can be mapped exactly by a simple transformation. For example, if the mesh
is periodic in the $x$-direction, the two edges must have nodes at exactly the
same height on each side. This can be easily accomplished using the
\lstinline[language=Bash]+create_aligned_mesh+ script in the scripts folder.

Alternatively, if you require a more complex periodic mesh with some structure between the periodic 
boundaries you can create one using \lstinline[language=Bash]{gmsh}. This can be achieved by 
setting up the periodic boundaries by using extrude and then deleting the 'internal' mesh.

The use of periodic domains requires additional configuration options. See
section \ref{Sect:periodic}.

\section{Meshing tools}
\index{mesh!meshing tools}

There are a number of meshing tools in both the scripts and tools directories.

\subsection{interval}
\index{mesh!meshing tools!interval}
This is a 1D mesh generator and is in the scripts directory. To use simply type:


\begin{lstlisting}[language = Bash]
`\fluiditysourcepath'/scripts/interval [options] left right name 
\end{lstlisting}

where left and right define the range of the line. It has a number of user defined input options:


\begin{center}
  \begin{tabular}{lp{.6\textwidth}}
    \hline
    \lstinline+--dx+ & constant interval spacing\\
    \lstinline+--variable_dx+ & interval spacing defined with a python function\\
    \lstinline+--region_ids+ & python function defining the region ID at each point\\
    \lstinline+--reverse+ & reverse order of mesh\\
    \hline
  \end{tabular}
\end{center}



\subsection{gmsh2triangle}
\index{mesh!meshing tools!gmsh2triangle}

This script converts ASCII Gmsh mesh files into triangle format. Whilst Fluidity
can read in Gmsh files directly as noted in section \ref{sect:using_gmsh}, in
cases where native Gmsh support does not work you should use 
\lstinline[language = Bash]+gmsh2triangle+ instead.

% Its use is now
% deprecated in favour of directly reading Gmsh files into \fluidity. See
% section \ref{sect:gmsh_format}.

\lstinline[language=Bash]{gmsh2triangle} stores the entire gmsh mesh in memory before writing it out to
triangle file. For very large meshes this is likely to be impractical,
and instead the \lstinline[language=Bash]{gmsh2triangle_large} script
should be used. This stores a minimal amount of data in memory at 
any one time, but is slower for small files as it requires the 
input .msh file to be read multiple times.

\lstinline[language=Bash]{gmsh2triangle[_large]} is used via:

\begin{lstlisting}[language = Bash]
gmsh2triangle[_large] input
\end{lstlisting}

where \lstinline[language = Bash]*input* is the input .msh file.

The \lstinline[language = Bash]+--2d+ flag can be used to instruct \lstinline+gmsh2triangle[_large]+
to process a 2D input .msh file. Otherwise, 3D input is assumed.

\subsection{triangle2vtu}
\index{mesh!meshing tools!triangle2vtu}
This converts triangle format files in to vtu format. It is in the fluidity tools directory and is used with:

\begin{lstlisting}[language = Bash]
`\fluiditysourcepath'/tools/triangle2vtu foo
\end{lstlisting}

Where \lstinline+foo+ is the triangle file base name (\lstinline+foo.node+ etc.)

\section{Decomposing meshes for parallel}
\label{decomp_meshes_parallel}
\index{parallel!mesh decomposition}

The first step in running a \fluidity\ set-up in parallel is to create the software
used to decompose the initial mesh into multiple parts. This can be made using:
\begin{lstlisting}[language=bash]
make fltools
\end{lstlisting}
inside your \fluidity\ folder. The following binaries will then be created in the \lstinline+bin/+ directory (see section \ref{sect:fltools}).
%You can then decompose the initial mesh the following command
%\begin{lstlisting}[language=bash]
%fluidity path /bin/fldecomp -n [PARTS] [BASENAME]
%\end{lstlisting}



\subsection{fldecomp}
\index{mesh!meshing tools!fldecomp}
\label{mesh!meshing tools!fldecomp}
This program is used to decompose a mesh into multiple regions, one per
process. In order to run fldecomp, if your mesh files have the base name
\lstinline{foo}\ and you want to decompose into four parts, type:
\begin{lstlisting}[language = Bash]
`\fluiditysourcepath'/bin/fldecomp -n 4 -m mesh_format mesh_file
\end{lstlisting}

Where:
\begin{center}
  \begin{tabular}{lp{.8\textwidth}}
    \lstinline+mesh_file+ & is the base name of your mesh file(s). For
    example, \lstinline+foo+ for \lstinline+foo.msh+ with Gmsh format, or
    \lstinline+foo.face/node/ele+ with triangle format.\\
    \lstinline+mesh_format+ & is the format of the mesh file you wish to
    decompose. It can take two values: \lstinline+gmsh+ or
    \lstinline+triangle+. If you omit the \lstinline+-m+ option,
    \lstinline+fldecomp+ will default to \lstinline+triangle+.
  \end{tabular}
\end{center}

For performance reasons, \lstinline[language=Bash]{fldecomp} supports only
binary Gmsh files. These are generated by passing the \lstinline{-bin}\
argument to Gmsh, for example:

\begin{lstlisting}[language=bash]
gmsh -3 -bin project.geo -o project.msh
\end{lstlisting}

This creates a 3D binary Gmsh mesh called \lstinline{project.msh} from the geometry file.



\subsection{flredecomp}
\index{mesh!meshing tools!flredecomp}
\label{mesh!meshing tools!flredecomp}
This is similar to fldecomp but runs in parallel. It is invoked as follows:
\begin{lstlisting}[language=bash]
mpiexec -n [target number of processors] \
   `\fluiditysourcepath'/bin/flredecomp \
        -i [input number of processors] \
        -o [target number of processors] [input flml] [output flml]
\end{lstlisting}

For example, to decompose the serial file \lstinline+foo.flml+
into four parts, type:

\begin{lstlisting}[language=bash]
mpiexec -n 4 `\fluiditysourcepath'/bin/flredecomp \
    -i 1 -o 4 foo foo_flredecomp
\end{lstlisting}

The output of running flredecomp is a series of mesh and vtu files as well
as the new flml; in this case \lstinline+foo_flredecomp.flml+.

%\subsubsection{Decomposing a periodic mesh}
\subsection{periodise}
\index{mesh!meshing tools!periodise}
\label{sect:decomposing_meshes_periodise}

To be able to run \fluidity\ on a periodic mesh in parallel you have to use
two tools (these tools are built as part of the fltools build target (see
section \ref{sect:fltools})):

\begin{itemize}
\item periodise
\item flredecomp (\ref{mesh!meshing tools!flredecomp})
\end{itemize}

The input to periodise is your flml (in this case
\lstinline{foo.flml}). This flml file should already contain the mapping for
the periodic boundary as described in section
\ref{Sect:periodic}. Periodise is run with the command:

\begin{lstlisting}[language=bash]
`\fluiditysourcepath'/bin/periodise foo.flml
\end{lstlisting}

The output is a new flml called \lstinline+foo_periodised.flml+ and the
periodic meshes. Next run flredecomp (section \ref{mesh!meshing
  tools!flredecomp}) to decompose the mesh for the number of processors
required. The flml output by flredecomp is then used to execute the actual simulation:

\begin{lstlisting}[language=bash]
mpiexec -n [number of processors] \
   `\fluiditysourcepath'/bin/fluidity [options] \
   foo_periodised_flredecomp.flml
\end{lstlisting}

\section{Pseudo-meshing}

The tool pseudo\_mesh enables the generation of a mesh of equal local resolution
to some prescribed input mesh. This is sometimes useful for the purposes of
interpolation comparison. pseudo\_mesh achieves this by computing an adaptivity
metric (see section \ref{sec:meshes_and_metrics}) derived from the input mesh
via a polar decomposition of the
elemental Jacobian \citep{micheletti2006}, and supplying this to the mesh adaptivity libraries
(see section \ref{sec:adaptive_remeshing_technology}).

pseudo\_mesh is not built by default as part of the \fluidity\ tools package. To
build the tool type:

\begin{lstlisting}[language = Bash]
cd tools
make ../bin/pseudo_mesh
\end{lstlisting}

pseudo\_mesh is invoked as:

\begin{lstlisting}[language = Bash]
pseudo_mesh [-thv] input_basename
\end{lstlisting}

where \lstinline[language = Bash]*input_basename* is the base name of an input
triangle file. If the \lstinline[language = Bash]*-t* flag is supplied then
then the metric used to form the output mesh is limited to target the element
count of the input mesh.

psuedo\_mesh is parallelised, and accepts the following options:

\begin{center}
  \begin{tabular}{lp{.6\textwidth}}
    \hline
 %   Flag & Function \\
 %   \hline
    \lstinline+-h+   & Display help \\
    \lstinline+-t+   & Limit the metric used to form the output mesh to
    target the element count of the input mesh \\
    \lstinline+-v+   & Verbose mode \\
    \hline
  \end{tabular}
\end{center}

\section{Mesh verification}
\index{mesh!verification}

The tool checkmesh can be used to form a number of verification tests on a mesh
in triangle mesh format. This tool is build as part of the fltools build target
(see section \ref{sect:fltools}), and is invoked as:

\begin{lstlisting}[language = Bash]
checkmesh input_basename
\end{lstlisting}

where \lstinline[language = Bash]*input_basename* is the base name of an input
triangle file. checkmesh tests for:

\begin{enumerate}
  \item Degenerate volume elements;
  \item Inverted tetrahedra;
  \item Degenerate surface elements;
  \item\label{item:mesh_tangling} Mesh tangling.
\end{enumerate}

% Test \ref{item:mesh_tangling} is performed by testing for any mesh self-intersections
% via mesh intersection (see section \ref{sec:supermeshing}).

checkmesh is parallelised. If running in parallel, it should be launched on
a number of processes equal to that in the mesh decomposition. In parallel
checkmesh output is written to \onlypdf\linebreak \lstinline+checkmesh.log-[process]+
and \lstinline+checkmesh.err-[process]+ log files.

\begin{example}
\begin{lstlisting}[language = Bash,keywordstyle=]
Checking volume elements for tangling ...
 In intersection_finder
 In advancing_front_intersection_finder
 Exiting advancing_front_intersection_finder
 Exiting intersection_finder
Tangled volume element found: 
Element: 1
Coordinates:
  0.10000000000000000E+001  0.00000000000000000E+000
  0.11666666666700001E+001  0.00000000000000000E+000
  0.83867056794499995E+000  0.54463903501499999E+000
Numbering:
           1
          14
           2
\end{lstlisting}
\caption{checkmesh reporting a mesh tangling error.}
\end{example}

\section{Non-\fluidity\ tools}

In addition to the tools and capabilities of \fluidity, there are numerous
tools and software packages available for mesh generation. Here, we describe 
two of the tools commonly used.

\subsection{Terreno}
\index{mesh!meshing tools!Terreno}
\index{Terreno}

Terreno uses a 2D anisotropic mesh optimisation algorithm to explicitly optimise for 
element quality and bathymetric approximation while minimising the number of mesh
elements created. The shoreline used in the mesh generation process is the result 
of a polyline approximation algorithm that where the minimum length of the resulting 
edges is considered as well as the distance an edge is from a vertex on the original 
shoreline segment being approximated. The underlying philosophy is that meshing and 
approximation should be error driven and should minimise user intervention. The 
latter point is two pronged: usability is paramount and the user should not need 
to be an expert in mesh generation to generate high quality meshes for their ocean 
model; fundamentally there must be clearly defined objectives to the mesh generation 
process to ensure reproducibility of results. The result is an unstructured mesh, 
which may be anisotropic, which focuses resolution where it is required to optimally 
approximate the bathymetry of the domain. The criterion to judge the quality of the 
mesh is defined in terms of clearly defined objectives. An important feature of the 
approach is that it facilitates multi-objective mesh optimisation. This allows one to 
simultaneously optimise the approximation to other variables in addition to the 
bathymetry on the same mesh, such as back-scatter data from soundings, material 
properties or climatology data. 

See the \href{http://amcg.ese.ic.ac.uk/terreno}{Terreno website}\ for more information.

\subsection{Gmsh}
\index{mesh!meshing tools!gmsh}
\index{gmsh}
\label{sect:meshing_tools_non_fluidity_gmsh}

Gmsh is a 3D finite element mesh generator with a build-in CAD engine and post-processor.
Its design goal is to provide a fast, light and user-friendly meshing tool with parametric
input and advanced visualisation capabilities. Gmsh is built around four modules: geometry, 
mesh, solver and post-processing. The specification of any input to these modules is done
either interactively using the graphical user interface or in ASCII text files using Gmsh's
own scripting language. 

For more information see the \href{http://geuz.org/gmsh/}{Gmsh website}\ or the \href{http://amcg.ese.ic.ac.uk}{AMCG
website}. An online manual is available at \href{http://geuz.org/gmsh/doc/texinfo/gmsh.html}{geuz.org/gmsh/doc/texinfo/gmsh.html}.

\subsection{Importing contours from bathymetric data into Gmsh}
\index{mesh!generation!gmsh!entering! bathymetry! data! into! fluidity! using! Gmsh}
\index{Entering bathymetry data into fluidity using Gmsh}

Gmsh can be used to create a mesh of a `real' ocean domain for use with \fluidity. An online guide to using Gmsh's built in
GSHHS plug-in is available at \href{http://perso.uclouvain.be/jonathan.lambrechts/gmsh_ocean/}{gmsh\_ocean}.
It is also possible to import contours from arbitrary bathymetry data sources into Gmsh. A guide and sample code detailing this process will
in the future be available on the \href{http://amcg.ese.ic.ac.uk}{AMCG website}.
